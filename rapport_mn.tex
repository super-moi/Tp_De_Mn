\documentclass[11pt]{article}
%Gummi|065|=)
\usepackage{graphicx}
\usepackage[utf8]{inputenc}
\usepackage[top=2cm, bottom=2cm, left=2cm, right=2cm]{geometry}
\usepackage{fancyhdr}
\usepackage{lastpage}
\usepackage{verbatim}
\usepackage{graphicx}
\usepackage{float}
\usepackage{wrapfig}
\usepackage{amssymb,amsmath}

\title{\textbf{Rapport - Modélisation et Simulation de l'Excitation Cardiaque}}
\author{Hadrien Titeux \& Thomas Ghesquiere}

\newcounter{question_num}
\setcounter{question_num}{1}


\date{}
\begin{document}

\maketitle

\section{Modélisation et discrétisation}
	\paragraph{Question \arabic{question_num} \\}
	La fonction $ euler1 $ permet de caluler $ u_{n} $ au temps t suivant en utilisant la méthode d'Euler. Cette fonction fonctionne aussi pour des vecteurs.
	
	\stepcounter{question_num}
	\paragraph{Question \arabic{question_num} \\}
	Pour vérifier la fonction euler1, nous allons la tester avec le systéme suivant :
	\[
	\left\{
	\begin {array}{r c l}
	\frac{du(t)}{dt} =& -0.1 t \ u(t) & \forall t < 50 \\
	u_{0}=& 1
	\end {array}
	\right.
	\]
	La solution de ce systéme est :
	\[
	\forall t < 50, \ u(t)=e^{0.05 \ t^2}
	\]
	Nous pouvons donc comparer la courbe réelle à celle obtenue avec la méthodes d'euler et ainsi vérifier la fonction $euler1$.
	\begin{figure}[H]
	\begin{center}
		\includegraphics[width=8 cm]{question2.jpeg}
		\caption{ Graphique de comparaison}
	\end{center}
	\end{figure}
	
	Les deux courbes obtenues se superposent parfaitement. Il suffit de diminier le pas $dt$ pour ne plus pouvoir les distinguer. La fonction $euler1$ permet donc bien de résoudre les systémes du même type que celui-ci.
	
	\stepcounter{question_num}
	\paragraph{Question \arabic{question_num} \\}
 	titeux	

	\stepcounter{question_num}
	\paragraph{Question \arabic{question_num} \\}
	Nous allons donc résoudre le modéle cellulaire proposé avec la fonction $euler1$. Voila les courbes des solutions approchées :
	\begin{figure}[H]
	\begin{center}
		\includegraphics[width=8 cm]{Solution0D.jpeg}
		\caption{ Solution approchée du modéle 0D }
	\end{center}
	\end{figure}
	
\section{Simulation en dimension supérieure}
	\stepcounter{question_num}
	\paragraph{Question \arabic{question_num} \\}
	La fonction $gen\_grille$ permet de créer une grille $ \Omega $ où chaque point contient les valeurs de e et r demandées.


	\stepcounter{question_num}
	\paragraph{Question \arabic{question_num} \\}
	Cette fonction construit une matrice de taille $n^2*n^2$ comme définie dans le sujet. Pour cette fonction, nous avons choisi de construire les diagonales exterieurs puis de compléter la matrice en construisant les "blocs" $T_{k}$ séparément.
	
	\stepcounter{question_num}
	\paragraph{Question \arabic{question_num} \\}
	Cette fonction utilise la fonction précédente pour réduire la taille de la matrice creuse créée. Nous avons mesuré le temps d'execution d'une multiplication entre la matrice et un vecteur. Nous obtenons des résultats trés différents entre L et Ls :
	\begin{figure}[H]
	\begin{center}
		\includegraphics[width=8 cm]{Comparaison_L_Ls.jpeg}
		\caption{ Comparaison des temps d'exécutions de L et Ls}
	\end{center}
	\end{figure}
	
	Pour Ls, les calculs sont trés rapides (presque instantanés). Mais pour Ls, les calculs sont exponentiels. En effet, pour Ls, la taille augmente de façon linéaire alors que pour L la taille augmente en $n^{4}$.
	
	\stepcounter{question_num}
	\paragraph{Question \arabic{question_num} \\}
	
	\stepcounter{question_num}
	\paragraph{Question \arabic{question_num} \\}
	
	\stepcounter{question_num}
	\paragraph{Question \arabic{question_num} \\}
	
	\stepcounter{question_num}
	\paragraph{Question \arabic{question_num} \\}



\section{Une méthode d’ordre supérieur }

		
			
\end{document}
